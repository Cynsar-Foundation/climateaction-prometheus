\documentclass[11pt]{article}

\usepackage{sectsty}
\usepackage{graphicx}

% Margins
\topmargin=-0.45in
\evensidemargin=0in
\oddsidemargin=0in
\textwidth=6.5in
\textheight=9.0in
\headsep=0.25in

\title{ Climate action through different perspectives: How peer to peer tech stack could address important issues. }
\author{ Saransh Sharma, Muellners Foundation }
\date{\today}

\begin{document}
\maketitle
\begin{abstract}

The current climate action that opens up several issues at large that are present in different applications of science, hence they require multiple level of thoughts opinions and experiment at abstraction to process and understand. We are aiming to reduce down habits through various financial incentives that may or may not happen right away. Proposed solutions are still far away from the reality where on ground activities linked from throwing plastic to jet fuels are still at present being funded by current economic system. Hence most research are focused on the idea that if we change the current financial system by adding the environmental cost  we could probably help in sustaining a natural pattern of climate change. We present a comprehensive document/ research where we show how current economic system displace good behaviour through resource allocation of central parties / agencies and hence moral hazard. This paper is quite critical about the behaviour mapping of the human  activities. We show these by demonstrating live surveys, in the last decade draw inference that if we really want to solve climate change the demand needs to be created. Similarly we propose a new set of technology tools for communities that are working on ground and would want to use these alternative systems to help address the crisis.

%Blockchains can be used as ledger systems that are verifiable, auditable, transparent, and run on a network of personal computers. Although best known as the computational underpinning of Bitcoin, they have applications beyond digital currency. They could transform contexts where verification of transactions without a central authority is currently a stumbling block. Among the applications being explored are so-called ‘smart contracts’, reputation systems, intellectual property, asset registers, voting, and the administration of decentralized organizations . There are also applications with environmental relevance. These include land claims registration, especially in regions where property ownership frequently is contested or where administrative institutions are not trusted, including species-rich parts of the developing world; establishment of alternative currencies backed by renewable energy generation, carbon mitigation, and sustainable innovation supply chain traceability to improve transparency and undermine corruption, particularly in agriculture and fishing; and tracking of illegal wildlife trade.
\end{abstract} 

\maketitle	
\pagebreak

% Optional TOC
% \tableofcontents
% \pagebreak

%--Paper--

\section{Introduction}

This is a comprehensive survey about multiple disciplines of science , we present how current economic structure is flawed at various level and provides zero safety to those who protect or preserve current resources. In the paper we define certain ideas like we use "resources" as economic resources unless specified like "natural resources". The outline of the paper, starts with how from last several years due to mass production and the need of scalability we have reduced natural resources by mixing it with practices that are clearly destroying or impacting the natural resources for clear profit optimisation , we don't intend to indemnify any person/individual or entities. Post this we state that how International law is flawed and only favours who have created the law, Law is definitely a pressing argument in our current study , we validate it multiple times through examples and experiments that we have conducted. Then we place an emphasis on how current academia lacks the strength to solve something complicated because of "trust" or we could perhaps say collaboration. Most academia is plagued by the idea of creating intellectual property style living, do something different get it patented and enjoy rent. The rent seeking behaviour is typically only enjoyed by few , since this leads to concentration of power. Post this we move how society fabrics and media narrative are placed to displace the facts that may or may move attitudes of people. Hence perspective management , then we clearly show that how right now in the name of solving climate crisis , societies are forming groups that again advancing in their self interest, like creating a business unit out of climate science/work. 

Financial marketplaces have zero value for counter-effect for the extraction of the resources, then we form an argument about the fact that how small experiments help us understand macro-perspectives. We ideate on how an alternative form of organisation with single goal of solving climate crisis through various steps like Changing Behavioural Shifts, Providing Long terms and short term financial incentives, providing people support on equitable manner sometimes referred as universal basic income.


\pagebreak
\section{Section 2}
Lorem Ipsum \\

%--/Paper--

\end{document}
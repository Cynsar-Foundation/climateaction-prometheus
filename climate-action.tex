\documentclass[11pt]{article}

\usepackage{sectsty}
\usepackage{graphicx}

% Margins
\topmargin=-0.45in
\evensidemargin=0in
\oddsidemargin=0in
\textwidth=6.5in
\textheight=9.0in
\headsep=0.25in

\title{ Climate action resource optimisation: How peer to peer tech stack could address important issues. }
\author{ Saransh Sharma, Muellners Foundation }
\date{\today}

\begin{document}
\maketitle
\begin{abstract}

The current climate action opens up several issues at large that are present in different science applications. Hence they require multiple levels of thoughts, opinions, and experiments at abstraction to process and understand. We aim to reduce down the effects of various agencies through various financial incentives that may or may not be effective right away. Proposed solutions are still far from the reality where the current economic system is still funding on-ground activities linked from throwing plastic to jet fuels. Therefore, most research focused on the idea that if we change the current financial system by adding the environmental cost, we could probably help sustain a natural climate change pattern. We present a comprehensive document/ research showing how current economic system displace good behaviour through resource allocation of major parties/agencies and hence moral hazard. This paper is quite critical of the behaviour mapping of human activities. We show these by demonstrating live surveys in the last decade to infer that the demand justified has to be created if we want to solve climate change. Similarly, we propose a new set of technology tools for communities working on the ground and would want to use these alternative systems to help address the crisis.


%Blockchains can be used as ledger systems that are verifiable, auditable, transparent, and run on a network of personal computers. Although best known as the computational underpinning of Bitcoin, they have applications beyond digital currency. They could transform contexts where verification of transactions without a central authority is currently a stumbling block. Among the applications being explored are so-called ‘smart contracts’, reputation systems, intellectual property, asset registers, voting, and the administration of decentralized organizations . There are also applications with environmental relevance. These include land claims registration, especially in regions where property ownership frequently is contested or where administrative institutions are not trusted, including species-rich parts of the developing world; establishment of alternative currencies backed by renewable energy generation, carbon mitigation, and sustainable innovation supply chain traceability to improve transparency and undermine corruption, particularly in agriculture and fishing; and tracking of illegal wildlife trade.
\end{abstract} 

\maketitle	
\pagebreak

% Optional TOC
% \tableofcontents
% \pagebreak

%--Paper--

\section{Introduction}

This paper is a comprehensive survey of multiple science disciplines; we present how the current economic and financial structure is flawed at various levels and provides zero safety to those who protect or preserve current resources. The paper defines specific ideas like using "resources" as economic resources unless specified like "natural resources". The paper's outline presents a strong argument in the name of demand, and industries are not attaching the environmental costs in production. We show how Capital and labour markets could yield benefits by adding these functions. Post this. We state how International Law supports suppressing environmental costs due to its non-binding nature. We validate it multiple times through examples and experiments that we have conducted. Then we emphasise how current academia lacks the strength to solve the current issue, since academia cooperate with corporation , or we could perhaps say wrong coordination. We demonstrate why rent-seeking behaviour in the economy is helping in an acceleration of inequality and financial instability and thus transferring resources to those who have no knowledge in assessment and care. Post this, and we move how society fabrics and media narratives placed to displace the facts that may or may move people's attitudes. Hence perspective management, then we clearly show that in solving the climate crisis, societies are forming groups that advance in their self-interest, like creating a business unit out of climate science/work.

This paper articulates the fact, why human induced gases are there in first place, its an effect due to industrialisation that took place in first phase or post war era where the factories that were producing machine guns are now producing fabrics or clothes. The resource extraction policy inhibit the land force use or labour force by criminalising the entire group or forcing them to move away from natural flora and fauna. Like in the case of several tribes every year displaced. Government policy should protect these displacement, but government complements this by providing these tribals who were once preserving the resources as obsolete and shifts them to a new habitat. This displacement is known everywhere and is observed and recorded, see the [] age 

Financial marketplaces have zero value for counter-effect for the extraction of the resources; then, we form an argument about how small experiments help us understand macro-perspectives. We ideate how an alternative form of governance with the single goal of fixing an economic system could prevent crisis through various steps like Changing Behavioural Shifts, Providing Long Term and short term financial incentives, and providing people support equitably, sometimes referred to as universal basic income. 





\pagebreak
\section{Carbon Markets}

European union leading the charge in creating and development of 
 \\

%--/Paper--

\end{document}